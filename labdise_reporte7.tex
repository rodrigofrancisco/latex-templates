\documentclass{mylib/reporteConCalif}
\usepackage{float}

\title{Reporte}
\author{rodrigofranciscopablo }

\subject{Laboratorio de Diseño Digital}
\mytitle{Reporte de práctica 7}
\mysubTitle{Decodificador BCD a 7 segmentos}
\students{Francisco Pablo \textsc{Rodrigo}}
\teacher{M.I. Guevara Rodríguez \textsc{Ma. del Socorro}}
\group{6}
\deliverDate{8 de abril de 2019}

\newcommand{\insertImage}[3]{
	\begin{figure}[H]
		\centering
		\includegraphics[width=#3cm]{#1}
		\caption{#2}
	\end{figure}
}

\begin{document}

\coverPage

\section{Objetivos}

\subsection{General}

El alumno diseñará circuitos combinacionales (mediana escala de integración).

\subsection{Particular}

El alumno analizará, diseñará e implementará un decodificador para display de 7 segmentos.

\section{Introducción}

Un decodificador o descodificador es un circuito combinacional, cuya función es inversa a la del codificador, es decir, convierte un código binario de entrada (natural, BCD, etc.) de N bits de entrada y M líneas de salida (N puede ser cualquier entero y M es un entero menor o igual a $2^N$), tales que cada línea de salida será activada para una sola de las combinaciones posibles de entrada.

Además, es un elemento digital que funciona a base de estados lógicos, con los cuales indica una salida determinada basándose en un dato de entrada característico, para el caso del decodificador de BCD a 7 segmentos,su función operacional se basa en la introducción a sus entradas de un número en código binario correspondiente a su equivalente en decimal para mostrar en los siete pines de salida establecidos para el integrado, una serie de estados lógicos que están diseñados para conectarse a un elemento alfanumérico en el que se visualizará el número introducido en las entradas del decodificador.

\insertImage{img/labdise_pract7/decoder_bcd}{Simple bozquejo de como conectar el decodificador a un display de 7 segmentos}{15}

\newpage
\section{Previo}

\newpage
\section{Desarrollo}



\section{Conclusiones}



\end{document}