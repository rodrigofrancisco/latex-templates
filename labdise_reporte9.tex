\documentclass{mylib/reporteConCalif}
\graphicspath{ {img/labdise_pract9/} }
\usepackage{float}

\title{Reporte}
\author{rodrigofranciscopablo }

\subject{Laboratorio de Diseño Digital}
\mytitle{Reporte de práctica 9}
\mysubTitle{Multiplexor}
\students{Francisco Pablo \textsc{Rodrigo}}
\teacher{M.I. Guevara Rodríguez \textsc{Ma. del Socorro}}
\group{6}
\deliverDate{23 de abril de 2019}

\begin{document}

\coverPage

\section{Objetivos}

\subsection{General}

El alumno diseñará circuitos combinacionales.

\subsection{Particular}

El alumno analizará, diseñará e implementará funciones utilizando multiplexores.

\section{Introducción}

Los multiplexores son circuitos combinacionales con varias entradas y una única salida de datos. Están dotados de entradas de control capaces de seleccionar una, y solo una, de las entradas de datos para permitir su transmisión desde la entrada seleccionada hacia dicha salida.

Ejemplo: Si utiliza un multiplexor de 4 canales de entrada. Una de los cuatro canales de entrada será escogido para pasar a la salida y ésto se logra con ayuda de las señales de control o selección.

La cantidad de líneas de control que debe de tener el multiplexor depende del número de canales de entrada. En este caso, se tiene que

numero de canales de entrada = $2^n$, en donde n es el número de líneas de selección.

%\insertImage{multiplexor}{Un multiplexor}{12}

\newpage
\section{Previo}

\newpage
\section{Desarrollo}

En esta práctica simulamos que queríamos encender una TV o un monitor o ambos para lo cual fue necesario ocupar el concepto de multiplexor, el cual para varias entradas nos regresa una sola salida, para este caso necesitamos 2 salidas, entonces implementamos 2 multiplexor de la siguiente manera.

\insertImage{1}{Implementación de multiplexores}{14}

Cada multiplexor fue hecho con código VHDL como se puede apreciar en las siguientes imágenes. En la imágenes observamos que la implementación fue hecha aplicando la programación secuencial a la que la mayoría de nosostros estamos acostumbrados, y por ello es que utilizamos la sentencia \textit{PROCESS}.

\insertImage{2}{Multiplexor 4x1}{12}
\insertImage{3}{Multiplexor 2x1}{12}

Por último comprobamos que la implementación fuera correcta y para ello realizamos la simulación correspondiente.

\insertImage{4}{Simulación del circuito con multiplexores}{14}

Por último, cargamos nuestro circuito a la FPGA, para nuestro caso la BASYS (Spartan3E) y estas son algunas de las salidas que observamos.

\insertImage{0011}{Combinación: 0011, los dos apagados}{10}

\insertImage{0100}{Combinación: 0100, monitor prendido}{10}

\insertImage{1010}{Combinación: 1010, TV prendida}{10}

\insertImage{1100}{Combinación: 1100, TV prendida}{10}

\insertImage{1101}{Combinación: 1101, ambos prendidos}{10}

\section{Conclusiones}

En la práctica vimos cómo funcionaban los multiplexores, los cuales tienen varias entradas y solo una salida de datos, con las entradas de selección se eleige solo una entrada de datos la cual se encarga de transmitir los datos hasta la salida.

Fue interesante simular una manera de controlar que dispositivo prende en determinado circuito ya que aún es sencillo, esto puede ser aplicado en la vida real y es común pedir esto en algún proceso a ser automatizado.

\end{document}
