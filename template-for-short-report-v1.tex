\documentclass[12pt]{article}
\usepackage[utf8]{inputenc}
\usepackage[spanish]{babel}
\usepackage{graphicx}
\usepackage{anysize}
\usepackage{fancyhdr} 
\usepackage[export]{adjustbox}
\usepackage{titlesec}
\usepackage{listings}
\usepackage{xcolor}

\marginsize{2.2cm}{2.2cm}{1.2cm}{1.5cm}          % {left}{right}{up}{down}
\renewcommand{\familydefault}{\sfdefault}
\decimalpoint%

\setlength{\parindent}{0in}
\titleformat*{\section}{\large\bfseries}

% Para insert código
\definecolor{codegreen}{rgb}{0,0.6,0}
\definecolor{codegray}{rgb}{0.5,0.5,0.5}
\definecolor{codepurple}{rgb}{0.58,0,0.82}
\definecolor{backcolour}{rgb}{1,1,1}

\lstdefinestyle{mystyle}{
    backgroundcolor=\color{backcolour},   
    commentstyle=\color{codegreen},
    keywordstyle=\color{magenta},
    numberstyle=\tiny\color{codegray},
    stringstyle=\color{codepurple},
    basicstyle=\ttfamily\footnotesize,
    breakatwhitespace=false,         
    breaklines=true,                 
    captionpos=b,                    
    keepspaces=true,                 
    % numbers=left,                    
    % numbersep=5pt,                  
    showspaces=false,                
    showstringspaces=false,
    showtabs=false,                  
    tabsize=2
}

\lstset{style=mystyle}

\graphicspath{{assets/logos/}{investigacion3.assets/}}
\newcommand{\materia}{Recursos y necesidades de México}
\newcommand{\clave}{2080}
\newcommand{\profesor}{Ing. Eduardo Alejandro Hernández González}
\newcommand{\grupo}{5}
\newcommand{\semestre}{2021-1}

\newcommand{\alumno}{Francisco Pablo \textsc{Rodrigo}}

\newcommand{\actividad}{Investigación V}
\newcommand{\titulo}{Recursos y Necesidades de México: Un dilema}

\newcommand{\fechaEntrega}{28 de enero de 2021}

%%%%%%%%%%%%%%%%%%%% ENCABEZADO %%%%%%%%%%%%%%%%%%%%%%%%%%%%
\pagestyle{fancy}
\fancyhf{}
\renewcommand{\headrulewidth}{0pt}
% \setlength{\headsep}{0.3in}

\begin{document}

%%%%%%%%%%%%%%%%%%% DATOS MINI PORTADA %%%%%%%%%%%%%%%%%%%%%%%%
\begin{minipage}[t]{0.7\linewidth}
    \vspace{-1cm}
    % \large{\textbf{UNAM}} \large{\textbf{FI}} \\
    \large{\textbf{\materia}}\\
    \large{\textbf{Grupo \grupo}}\\
    % \large{\textbf{Profesor:} \profesor}\\ [1.5cm]
    \large{\textbf{Alumno:} \alumno} \\
    \textbf{Fecha:} \fechaEntrega%
\end{minipage}\hfill
\begin{minipage}[t]{0.2\linewidth}
    \vspace{-1.2cm}
    \begin{flushright}
        % \includegraphics[width=2cm]{fiblack}
        \includegraphics[width=2cm]{unam.jpg}\\
        % \vspace{10cm}
        \large{\semestre}    
    \end{flushright}
\end{minipage}
% \vspace{2mm}
%%%%%%%%%%%%%%%%%%% CONTENIDO %%%%%%%%%%%%%%%%%%%%%%%%
\begin{center}
\large{
    \textbf{\actividad}\\
    \textbf{\titulo} \\ 
}
\end{center}

Existen una conste lucha entre las necesidades y los recursos que provocan ina
de interrogante: ¿Los recursos de nuestro país son suficientes para cubrir las
necesidades del mismo? Cono todo lo aprendido de la geografía de México, los
climas y otros recursos naturales tal vez en primera instancia uno pensaría que
es fácil responder con una afirmación. Sin embargo, es correcto detenerse a
analizar más a fondo y observar la realidad de muchos estados de la república
como Oaxaca, Chiapas o Guerrero.\\

Por otra parte hay que dejar de pensar en 2D y avanzar hacia la siguiente
dimensión, los recursos, como hemos visto, no solo se tratan de lo que la
naturaleza ha dispuesto en la región en la que nos encontramos sino que va
muchísimo más allá. Involucra capital humano, tiempo, organización, etc.
Dicho lo anterior, se puede intentar explicar en que sentido va la contienda
entre recursos y necesidades, no es un tema de recursos naturales, mas tiene que
ver con recursos de otro tipo, organizaciones por ejemplo. De esta manera se
alcanza a ver el problema que tienen muchos estados. Su falta de organización
y/o planeación ha sido una de las causas principales de los problemas que
enfrenta no el estado, sino directamente la población: falta de educación,
vivienda, comida, etc.\\


\end{document}
