\documentclass{mylib/reporteConCalif}
\graphicspath{ {img/labdisp_pract5/} }

\title{Reporte}
\author{rodrigofranciscopablo }
\mytitle{Reporte de práctica 5}
\subject{Laboratorio de Dispositivos y circuitos electrónicos}
\mysubTitle{Circuitos recortador y multiplicador de tensión}
\students{Francisco Pablo \textsc{Rodrigo}}
\teacher{M.I. Guevara Rodríguez \textsc{Ma. del Socorro}}
\group{8}
\deliverDate{27 de marzo de 2019}

\begin{document}

\coverPage

\section{Objetivos}

\subsection{General}

Analizar y diseñar circuitos electrónicos que contienen diodos semiconductores.

\subsection{Particular}

Analizar, diseñar, simular e implementar circuitos recortador, sujetador y multiplicador de tensión utilizando diodos
de propósito general.

\section{Introducción}

\subsection{Circuito recortador}

Un limitador o recortador es un circuito que, mediante el uso de resistencias y diodos, permite eliminar tensiones que no nos interesen para que no lleguen a un determinado punto de un circuito. Mediante un limitador podemos conseguir que a un determinado circuito le lleguen únicamente tensiones positivas o solamente negativas.

Estos tipos de circuitos utilizan dispositivos de una o más uniones PN como elementos de conmutación. Se diseñan con el objetivo de recortar o eliminar una parte de la señal que se le introduce en sus terminales de entrada y permita que pase el resto de la forma de onda sin distorsión o con la menor distorsión posible. Para realizar esta función de recortar, los recortadores hacen uso de la variación brusca que experimenta la impedancia entre los terminales de los diodos y transistores al pasar de un estado a otro, de ahí que sean los elementos básicos en dichos circuitos. 

\insertImage{recpos}{Circuito recortador}{8}
	
\subsection{Circuito sujetador}

Estos circuitos basan su funcionamiento en la acción del diodo, pero al contrario que los limitadores no modificarán la forma de onda de la entrada, es decir su voltaje o tipo de corriente eléctrica, sino que le añaden a ésta un determinado nivel de corriente continua. Esto puede ser necesario cuando las variaciones de corriente alterna deben producirse en torno a un nivel concreto de corriente continua.

\insertImage{suj}{Rectificador de onda completa}{8}

\section{Circuito Multiplicador}

Un Multiplicador de tensión es un circuito eléctrico que convierte tensión desde una fuente de corriente alterna a otra de corriente continua de mayor voltaje mediante etapas de diodos y condensadores.


Un circuito multiplicador de voltaje es un arreglo de capacitores y diodos rectificadores que se utiliza con frecuencia para generar altos voltajes de Corriente Directa. Este tipo de circuito se utiliza el principio de la carga en paralelo de capacitores, a partir de la entrada de Corriente Alterna y añadiendo voltaje a través de ellos en serie se obtiene voltajes de CD más alto que el voltaje de la fuente. Circuitos individuales de multiplicadores de Voltaje (a menudo llamados etapas) se pueden conectar en serie para obtener aún más altos voltajes de salida.

\insertImage{mult}{Rectificador de onda completa}{8}


\newpage
\section{Previo}

\subsection{Diseña un circuito recortador positivo}

\insertImage{recPolPos}{Circuito recortador positivo a 6 VDC}{12}

\subsection{Diseña un circuito recortador negativo}

\insertImage{recPolNeg}{Circuito recortador negativo a - 6 VDC}{12}

\subsection{Diseña un circuito duplicador de tensión}

\insertImage{dup_1mf}{Circuito duplicador}{12}

\subsection{Diseña un circuito triplicador de tensión}

\insertImage{trip_10mf}{Circuito triplicador}{12}

\subsection{Diseña un circuito cuadriplicador de tensión}

\insertImage{cuad_10mf}{Circuito cuadriplicador}{12}

\newpage
\section{Desarrollo}

\subsection{Recortadores}

Para esta primera sección de la práctica comprobamos que el circuito de la imagen de abajo es capaz de recortar parte
de la señal de entrada. \\

Para el caso del recortador positivo observamos en el osciloscopio que la señal obtenida es la siguiente.

\insertImage{d_recneg}{Circuito recortador}{9}

Para el caso del recortador negativo observamos en el osciloscopio que la señal obtenida es la siguiente.

\insertImage{d_recpos}{Circuito recortador}{9}

\subsection{Multiplicadores de tensión}

Para el caso de los multiplicadores debemos de tener cuidado porque estos circuitos son capaces de duplicar, triplicar, ... la señal
de entrada. \textbf{De lo anterior se deriva que NO podemos usar el OSCILOSCOPIO para medir las señales de salida.} Entonces, a continuación
se presentan algunas fotografías de la lectura del multimetro.

Para el circuito duplicador y un transformador de 20 V, pero en el cual solo usamos el \textit{tab central} y una \textit{fase} del transformador. 
Es decir la señal de entrada son 10 V, obtenemos lo siguiente:

\insertImage{d_dup}{Circuito duplicador}{9}

El voltaje se eleva mediante etapas de diodos y capacitores. En la figura de abajo tenemos el diagrama de doblador de voltaje de media onda el cual está compuesto por una fuente de alimentación de 6 VAC, 2 capacitores de 12 uF y 2 diodos rectificadores.

\insertImage{explicacion_mult}{Circuito multiplicador}{10}

Cada etapa esta compuesta por un diodo y un capacitor. En el primer semiciclo de la onda de voltaje alterna el diodo D1 se polariza directamente y D2 de manera inversa, por lo tanto el capacitor C1 se carga con el voltaje de la fuente (Vc1 = V1).
Luego, durante el semiciclo negativo de la onda alterna de voltaje el diodo D1 se polariza invesamente y D2 se polariza directamente. Ya que el voltaje en el capacitor C1 no se puede descargar, el capacitor C2 se carga con el voltaje de la fuente más el voltaje en el capacitor C1 (Vc2 = V1+Vc1). Teniendo en cuenta que los diodos rectificadores producen una pérdida de voltaje al polarizarse directamente y reemplazando el valor de Vc1=V2 en Vc2=Vc1+V1 tenemos que el voltaje en el capacitor C2 es igual a Vc2= 2V1. Hay que tomar encuenta que los diodo producen una pérdida de voltaje dependiendo de su composición química (Germacio / Silicio) puede ser de 0.2V a 0.7V. Tomando en cuenta esto el voltaje Vc2 sería igual a Vc2 = 2V1-2Vdiodo.

\textit{Algo similar pasaría pasaría para un circuito triplicador y para un cuadriplicador}

\section{Conclusiones}

Es increíble comprobar que el diodo tiene infinitas aplicaciones. Dichas 
aplicaciones dependerán de como el diodo esta dispuesto en el circuito y de los componentes que lo acompañan.
Por ejemplo, observamos que si el diodo esta acompañado por una fuente de voltaje a corriente directa y a la vez el diodo esta polarizado de forma positiva obtenemos un \textit{recortador positivo}.

Por otra parte, si tenemos un circuito con capacitores y diodos, dependiendo de como lo conectemos y del número de diodos y capacitores podríamos obtener un duplicador o triplicador de voltaje.

Aprendimos que si trabajamos con corriente continua no importa la polarización del capacitor ya que la corriente continua esta cambiando de signo a 60 ciclos por cada segundo. Sin embargo, cuando se intenta conectar un capacitor después de un diodo tenemos que ser muy conscientes de la polarización ya que si lo conectamos mal explotará.

\end{document}