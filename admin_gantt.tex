\documentclass{mylib/reporteCorto}
\graphicspath{ {img/admin/} }

\usepackage{float}
\usepackage{multirow}
\title{Reporte}
\author{rodrigofranciscopablo }

\subject{Administración de proyectos de Software}
\mytitle{Diagramas de Gantt}
\mysubTitle{Trabajo de investigación}
\students{Francisco Pablo \textsc{Rodrigo}}
\teacher{Ing. Gamboa Beltran \textsc{Jonahan}}
\group{1}
\deliverDate{8 de abril de 2019}

\begin{document}

\coverPage

% reporte de investigación sobre el diagrama de Gant, con antecedentes, origen, descripción metodólogica y un ejemplo de aplicación

\tableofcontents

\section{Antecedentes y origen}

En 1896 Karol Adamiecki desarrolló un medio de representación gráfica de procesos interdependientes con el fin de mejorar la visibilidad de los programas de producción. Dada la posición de cada tarea a lo largo del tiempo hace que se puedan identificar las relaciones e interdependencias.

En 1931 se publicó un artículo conocido más ampliamente para describir el diagrama de Karol Adamiecki, quien lo llamó el harmonogram o harmonograf. Adamiecki, sin embargo, publicó sus obras en polaco y ruso, idiomas poco conocidos en el mundo de habla inglesa. En ese momento, un método similar se había popularizado en occidente por Henry Gantt (quien había publicado artículos sobre el mismo en 1910 y 1915). Con modificaciones menores, la carta de Adamiecki es ahora más comúnmente conocida en inglés como el diagrama de Gantt, ya que fue Henry Laurence Gantt quien, entre 1910 y 1915, modificó y divulgó este tipo de diagrama en occidente.

\section{¿Qué es un diagrama de Gantt?}

Un diagrama de Gantt no es más que un gráfico de barras, una estrategia de planeación que puede servir como una especie de guía a la hora de poner en marcha todas las labores necesarias para ir de un punto A hasta un punto B. En este diagrama las acitividade sdel royecto siempre se muestran de manera vertical, mientralsa que los tiempos aparecen representados de manera horizontal.

\insertImage{gantt1}{Ejemplo de diagrama de Gantt}{15}

\section{Metodología}

Generalmente se trabaja con dos conjuntos de columnas, uno de ellos se encuentra ubicado del lado derecho de la tabla y el otro del lado izquierdo.

Las columnas que se encuentran del lado izquierdo contienen información sobre las actividades a realizar, el material, el personal y demás recursos requeridos para la ejecución del proyecto; mientras que las columnas que se encuentran del lado derecho reflejan el tiempo requerido para realizar cada una de las actividades, el cual puede corresponder a días, semanas o meses, según sea más conveniente.

Para crear una gráfica de Gantt, los pasos básicos que debemos seguir son los siguientes:Para crear una gráfica de Gantt, los pasos básicos que debemos seguir son los siguientes.

\begin{enumerate}
  \item
    Lo primero que debemos hacer es dibujar las columnas y filas que formarán parte de nuestra gráfica, y en las cuales iremos colocando toda la información que consideremos necesaria. El número de columnas y filas dependerá de la cantidad de datos que se quiera incluir en el gráfico, pero por lo general puede trabajarse con un conjunto sencillo de 4 columnas ubicadas al lado izquierdo de la tabla, seguido del número de columnas que sea necesario para apuntar la cantidad de días, semanas o meses que tomará la ejecución del proyecto. Por otro lado, el número de filas que dibujaremos dependerá de la cantidad de actividades involucradas en el plan.

    \item
    En las filas de la primera columna colocaremos todas las actividades que sea necesario realizar, una debajo de la otra, cada una en una cuadrícula distinta.

    \insertImage{gantt2}{En el lado izquierdo de la gráfica de Gantt, se procede a colocar cada una de las actividades en la primera columna.}{12}

    \item
    En las filas de la segunda columna colocaremos la fecha estimada de inicio de cada una de las actividades.

    \item
    En la tercera columna apuntaremos la fecha estimada de finalización o cierre de cada tarea.

    \insertImage{gantt3}{El lado derecho de la gráfica funciona como una especie de calendario.}{12}

    \item
    En la cuarta columna colocaremos el tiempo total estimado para la duración de cada una de las actividades, teniendo en cuenta la unidad de tiempo seleccionada, ya sean días, semanas o meses. Es muy importante considerar que, aunque no sepamos con total certeza la cantidad de tiempo que demorará una actividad, debemos colocar parámetros razonables a los cuales podamos apegarnos para administrar mejor nuestro tiempo y nuestro esfuerzo.

    \item
    Lo que haremos a continuación será dibujar un conjunto de columnas que irá del lado derecho de la tabla, justo después de la cuarta columna en la que ubicamos la duración total de las actividades. El número de estas columnas dependerá de la cantidad de tiempo necesaria para culminar el proyecto en su totalidad, y estarán dividas igualmente en función de la unidad de tiempo con la que se haya decidido trabajar (días, semanas o meses). En las filas de estas columnas iremos dibujando bloques de color que representarán el tiempo de duración de cada actividad, lo que nos ayudará a determinar si existe alguna prelación en el desarrollo de las mismas.

    \item

    El siguiente paso será demarcar el momento actual del proyecto, para lo cual trazaremos una línea vertical que partirá del día, semana o mes correspondiente a la fecha presente, y rellenaremos con un tono más oscuro las celdas de los días, semanas o meses que hayan transcurrido. Además, mediremos el progreso de cada una de las actividades dibujando una línea oscura y fina dentro de los bloques de color que determinan la duración de cada tarea. Si la actividad ha culminado, la línea de color oscuro tendrá la misma longitud que el bloque de color correspondiente a su duración, mientras que si aún se encuentra en proceso, tendrá la longitud que corresponda a su avance o progresión en el tiempo.

    \insertImage{gantt4}{Los bloques de color azul de la gráfica de Gantt indican la duración de cada tarea de una forma mucho más gráfica, y nos permiten conocer a simple vista si existe alguna relación de dependencia entre una y otra actividad.}{12}

\end{enumerate}
\section{Referencias y Bibliografía}
%https://es.wikipedia.org/wiki/Diagrama_de_Gantt
\begin{itemize}
  \item http://tugimnasiacerebral.com/herramientas-de-estudio/\\que-es-un-diagrama-o-grafica-de-gantt
\end{itemize}


\end{document}
