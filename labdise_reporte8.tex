\documentclass{mylib/reporteConCalif}
\graphicspath{ {img/labdise_pract8/} }
\usepackage{float}

\title{Reporte}
\author{rodrigofranciscopablo }

\subject{Laboratorio de Diseño Digital}
\mytitle{Reporte de práctica 8}
\mysubTitle{Decodificador utilizando mediana escala de integración}
\students{Francisco Pablo \textsc{Rodrigo}}
\teacher{M.I. Guevara Rodríguez \textsc{Ma. del Socorro}}
\group{6}
\deliverDate{23 de abril de 2019}

\begin{document}

\coverPage

\section{Objetivos}

\subsection{General}

El alumno diseñará circuitos combinacionales.

\subsection{Particular}

El alumno analizará, diseñará e implementará decodificadores en mediana escala de integración.
\section{Introducción}

Un decodificador es un circuito integrado que genera todos los minitérminos correspondientes a n entradas.Cada salida corresponde a un minitérmino, empezando con la salida superior que corresponde al minitérmino 0. Como se muestra en la siguiente figura.

\insertImage{dec1}{Ejemplo de decodificador}{4}

En la practica, la mayoría de los decodificadores tienen las salidas complementadas indicándolo mediante una pequeña burbuja o circulo en la salida, es decir, el decodificador genera maxitérminos(0) en lugar de minitérminos(1).

Una de las aplicaciones de un decodificador es la implementación de funciones Booleanas. Una función Booleana de n variables puede ser implementada fácilmente al unir los minitérminos(maxitérminos) correspondientes a la función utilizando una compuerta OR(NAND).


\newpage
\section{Previo}

\newpage
\section{Desarrollo}


\section{Conclusiones}


\end{document}
