\documentclass[12pt]{book}
\usepackage[T1]{fontenc}                    % UTF-8 output
\usepackage[utf8]{inputenc}                 % Use UTF-8 codification
\usepackage[spanish]{babel}                 % Get symbols and hyphenation for
                                            %spanish

\usepackage{graphicx}                       % Add images or other graphics
\usepackage{geometry}                       % For margins

\usepackage{fancyhdr}                       % Style header and footer
\usepackage{array}                          % Align fix size columns in tables
\usepackage[export]{adjustbox}              % Fix table sizes

\usepackage{marginnote}                     % Adds notes at the margins
\usepackage{float}                          % For better placement figures
\usepackage{titlesec}                       % Style sections and subsections
\usepackage{longtable}                      % Create page splitable tables
\usepackage{multicol}                       % Multicols library

\usepackage{tikz}                           % Create figures 
\usepackage[framemethod=TikZ]{mdframed}     % Frames and border boxes
\usepackage{enumitem}                       % Changes item style
\usepackage{environ}                        % Creating environments that have
                                            % issues with parentesis

\usepackage{scrextend}                      % For pages and NewEnviron command
\usepackage[strict]{changepage}             % to get the right page 

\usepackage{xcolor}
\usepackage{amsmath}

% JUST FOR TESTING
\usepackage{lipsum}

% Adjusting book margins
\addtolength{\topmargin}{-8mm}
\addtolength{\textheight}{20mm}
\addtolength{\textwidth}{-18mm}
\addtolength{\marginparwidth}{14mm}
\addtolength{\marginparsep}{4mm}
\addtolength{\evensidemargin}{18mm}
\addtolength{\oddsidemargin}{-6mm}
\setlength{\parindent}{0in}                 % Not left margin at each paragraph 
\setlength{\headsep}{0.5in}                 % Changing headsep length
                                            % headsep is the vertical length 
                                            % between header an text area
\addtolength{\headwidth}{115pt}             % Increase header length
\decimalpoint%                              % Using dot instead of comma

% Defining some colors
\definecolor{darkblue}{RGB}{0, 0, 102}
\definecolor{gray75}{gray}{0.75}

% 20pt horizontal space
\newcommand{\hsp}{\hspace{20pt}}

\renewcommand{\familydefault}{\sfdefault}   % Changing font family 
% Redefine header chapter
\renewcommand{\chaptermark}[1]{\markboth{\MakeUppercase{\thechapter.\ #1}}{}}

% Style chapters title
\titleformat{\chapter}[hang]{\Huge\bfseries}
{\color{darkblue}\thechapter\hsp\textcolor{gray75}{|}\hsp}{0pt}
{\color{darkblue}\Huge\bfseries}

% Changing section style
\titleformat*{\section}
  {\color{darkblue}\LARGE\bfseries} 
% Changing subsection style
\titleformat*{\subsection}
  {\color{darkblue}\Large\bfseries}

% Defining column content alignment for fix size columns
\newcolumntype{L}[1]
  {>{\raggedright\let\newline\\\arraybackslash\hspace{0pt}}m{#1}}
\newcolumntype{C}[1]
  {>{\centering\let\newline\\\arraybackslash\hspace{0pt}}m{#1}}
\newcolumntype{R}[1]
  {>{\raggedleft\let\newline\\\arraybackslash\hspace{0pt}}m{#1}}  

% Style item as a circle
\newcommand*\circItem[1]{%
  \begin{tikzpicture}[baseline=(C.base)]
    %\node[draw,circle,inner sep=1pt,minimum size=3ex](C) {#1};
    \node[draw=darkblue,circle,inner sep=1pt,minimum size=3ex](C) {#1.};
  \end{tikzpicture}
}

\newcounter{activityCount}[chapter]
%TODO:- Improve if statement
\NewEnviron{wideBox}[1]{
  \refstepcounter{activityCount}
  \checkoddpage
  \ifoddpage
    \parbox[t]{#1\paperwidth}{
      \begin{mdframed}[
        frametitle={\color{darkblue}Actividad~\theactivityCount},
        frametitlerule=true,
        topline=false,
        bottomline=false,
        rightline=false,
        leftline=false,
        innerrightmargin=2pt,
        innerleftmargin=2pt,
      ]
      \BODY
      \end{mdframed} 
    }
  \else
    %TODO:- Get ride of hardcode value
    \begin{addmargin}[-3.5cm]{0em}
    \parbox[t]{#1\paperwidth}{
      \begin{mdframed}[
        frametitle={\color{darkblue}Actividad~\theactivityCount},
        frametitlerule=true,
        topline=false,
        bottomline=false,
        rightline=false,
        leftline=false,
        innerrightmargin=2pt,
        innerleftmargin=2pt,
      ]
      EVEN page
      \BODY
      \end{mdframed} 
    }
    \end{addmargin}
  \fi
}

\newcounter{definitionCount}[chapter]
\newenvironment{definition}{
  \refstepcounter{definitionCount}
  \begin{mdframed}[
    frametitle={\color{darkblue}Definición~\thedefinitionCount},
    frametitlerule=true,
    roundcorner=10pt
  ]
}{
  \end{mdframed} 
}


%%%%%%%%%%%%%%%%%%%%%%%%%%%%%%%%%%%%%%%%%%%%%%%%%%%%%%%%%%%%%%%%%%%%%%%%%%%%%%%
%%%%%%%%%%                        Header Style                       %%%%%%%%%%
%%%%%%%%%%%%%%%%%%%%%%%%%%%%%%%%%%%%%%%%%%%%%%%%%%%%%%%%%%%%%%%%%%%%%%%%%%%%%%%

\renewcommand{\headrulewidth}{0pt}
\pagestyle{fancy}
\fancyhead{}
\fancyhead[LE]{\nouppercase\leftmark}% LE -> Left part on Even pages
\fancyhead[RO]{\nouppercase\rightmark}% RO -> Right part on Odd pages

\begin{document}

%%%%%%%%%%%%%%%%%%%%%%%%%%%%%%%%%%%%%%%%%%%%%%%%%%%%%%%%%%%%%%%%%%%%%%%%%%%%%%%
%%%%%%%%%%                Variables definition                       %%%%%%%%%%
%%%%%%%%%%%%%%%%%%%%%%%%%%%%%%%%%%%%%%%%%%%%%%%%%%%%%%%%%%%%%%%%%%%%%%%%%%%%%%%

\graphicspath{ {latex/assets/} }
\newcommand{\materia}{Trigonometría}

%%%%%%%%%%%%%%%%%%%%%%%%%%%%%%%%%%%%%%%%%%%%%%%%%%%%%%%%%%%%%%%%%%%%%%%%%%%%%%%
%%%%%%%%%%                        Contents                           %%%%%%%%%%
%%%%%%%%%%%%%%%%%%%%%%%%%%%%%%%%%%%%%%%%%%%%%%%%%%%%%%%%%%%%%%%%%%%%%%%%%%%%%%%

\chapter{\materia}

\section{Ángulos}

\textit{Diseñe una carta ASM con hasta 16 estados, 4, entradas (X,Y,Z,W) Y
4 salidas (S0,S1,S2,S3) y determine la tabla de verdad por el método
de Direccionamiento por Entrada-Estado.}\\

\marginnote{This is a margin note using the geometry package, set at 3cm
vertical offset to the line it is typeseted.}

\subsection{Sistema sexagesimal}

\begin{wideBox}{0.8}
  \lipsum[1]
\end{wideBox}

\textit{Determine el número de bits de memoria que se ahorran al
implementar una carta ASM que posee 4 entradas (X,Y,Z,W), 20
estados, 8 salidas (S0-S7), mediante el método de ``direccionamiento
entrada-estado'' respecto al método ``direccionamiento por trayectoria''.}\\

Ahorro = ``direccionamiento por trayectoria'' - ``direccionamiento
entrada-estado''\\ 

%\begin{equation}
  %\text{Ahorra} = $(2^{\text{bits para edos+ entradas}})(bits de liga+salidas)$
%\end{equation}

\lipsum

\marginnote{This is a margin note using the geometry package, set at 3cm
vertical offset to the line it is typeseted.}[-2cm]

\begin{wideBox}{0.6}
  Convierte los siguientes ángulos a \textit{solo} grados
  \begin{multicols}{4}
    \begin{enumerate}[label=\protect\circItem{\arabic*}]
      \setlength\itemsep{4mm}
      \item $40^\circ 10' 15''$
      \item $61^\circ 42' 21''$
      \item $1^\circ 2' 3''$
      \item $73^\circ 40' 40''$
      \item $9^\circ 9' 9''$
      \item $98^\circ 22' 45''$
      \item $23^\circ 24' 25''$
      \item $320^\circ 20' 17''$
      \item $322^\circ 50' 59''$
      \item $9^\circ 2' 1''$
      \item $9^\circ 2' 1''$
      \item $9^\circ 2' 1''$
    \end{enumerate}
  \end{multicols}

  Convierte los siguientes ángulos a su equivalente en grados, minutos y
  segundos.
  \begin{multicols}{4}
    \begin{enumerate}[label=\protect\circItem{\arabic*}]
      \setlength\itemsep{4mm}
      \item $40.32^\circ$
      \item $61.24^\circ$
      \item $18.255^\circ$
      \item $29.411^\circ$
      \item $19.99^\circ$
      \item $44.01^\circ$
      \item $32.123^\circ$
      \item $34.98^\circ$
      \item $32.29^\circ$
      \item $32.29^\circ$
      \item $32.29^\circ$
      \item $769^\circ$
    \end{enumerate}
  \end{multicols}
\end{wideBox}

\lipsum

\section{Razones trigonométricas}

\lipsum[1-6]

\textit{\lipsum[8]}\\

\begin{wideBox}{0.8}
  \lipsum[3]
\end{wideBox}

Testing new environment

\begin{definition}
  \lipsum[3]
\end{definition}

\end{document}
