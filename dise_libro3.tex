\documentclass{mylib/reporte}
\usepackage{float}
\title{Reporte}
\author{rodrigofranciscopablo }

\subject{Diseño Digital Moderno}
\mytitle{Reporte Tercer Libro}
\mysubTitle{El tercer ojo}
\students{
	Francisco Pablo \textsc{Rodrigo}
}
\teacher{Ing. Mandujano Wild \textsc{Roberto F.}}
\group{6}
\deliverDate{1 de junio de 2019}

\begin{document}

\coverPage

\section{Introducción}

Este libro narra las experiencias vividas por el protagonista durante su infancia en el Tibet, su introducción en el mundo de los lamas y sus extraordinarias viviencias de iniciación por el tortuoso camino del aprendizaje. Un libro que definitivamente te cambia, interesantísimo y transportador hacia otro mundo.


\section{Reporte}

El libro titulado 'El tercer ojo' narra la vida de un monje médico tibetano, desde su infancia, hasta que su vida adulta. La vida de Lobsang Rampa resulta ser muy interesante porque desde el principio su vida lo prepararon para ser 'alguien destinado a hacer cosas grandes', la preparación incluyó grandes sacrificios y pruebas que tuvo que afrontar desde su muy corta edad. Por ejemplo, desde pequeño tuvo que soportar intensos entramientos de un arte marcial que se asemeja al 'yudo'. Además, tuvo aprender matemáticas, aunque no especifican de que tipo, para poder comprender astrología, la cual era considerada una ciencia en su país e inclusive con base en esta 'ciencia' se tomaban decisiones de todo tipo, políticas, agropecuarias, etc. \\

En el mundo occidental sabemos muy poco de la cultura oriental y sabemos todavía menos sobre la cultura de Tibet, básicamente todo lo que sabemos es por las películas que hacen referencia a ellos y puede que no sean ideas del todo correcto, a medida que el personaje cuenta sobre su vida va desmintiendo algunos mitos que nos hemos creado sobre \textit{los monjes tibetanos}.\\
Por ejemplo, sabemos que los lamas se caracterizan por ser gente pacifica y esto en parte es cierto, siempre intentan estar en armonía con todo aquello que los rodea, sin embargo a la hora de pelear son verdaderamente buenos, pues las artes marciales están completamente inmersas en su forma de vida.\\

Otra cosa que cabe destacar es que su modo de vida, que de este lado del mundo nos parecería \textit{incivilizado}, con solo decir que su combustible principal, su alimentación principal es se basa en sampa que es una antigua receta originaria del Tíbet de bolitas de granos y cereales tostados y molidos y amalgamados con miel, altamente energéticas y super alimenticias. Lo más curioso de este recondito país es que \textbf{no existen las ruedas } en tibet, porque los animales resultan indispensables para poder transportarse.
Por otra parte, la mensajería la hacen personas destinadas a ello y se comprometen tanto que siempre llegan exahustos a comunicar el mensaje de manera oral, es más si no llegan cansados se considera que no hicieron bin su trabajo.


%\section{Conclusiones}




\end{document}
