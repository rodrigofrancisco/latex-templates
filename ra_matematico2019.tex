\documentclass[a4paper]{mylib/examenStudent}

\org{LUDOMATICS}

\context{Curso de Ingreso a Media Superior}
\subject{Razonamiento Matemático}
\myexam{Ejercicios en clase}

\setlength{\columnsep}{1cm}

\begin{document}
\coverPage
\begin{multicols*}{2}

	\section*{Sucesiones númericas}
	
	Determinar los números que faltan en la siguiente sucesión númerica.

	\pregunta{7, 14, 16, 32, \rule{1cm}{0.4pt}, 68 }{null}
	\opciones{34}
		{48}
		{42}
		{64}	
	\idpreg{2016-manIPN-rmat-act237-a}{34}
	
	\pregunta{4, 4, 8, 24, \rule{1cm}{0.4pt} }{null}
	\opciones{12}
		{70}
		{80}
		{96}	
	\idpreg{2016-manIPN-rmat-act237-b}{96}

	\pregunta{2, 5, 8, 11, \rule{1cm}{0.4pt} }{null}
	\opciones{12}
		{13}
		{14}
		{15}	
	\idpreg{2016-manIPN-rmat-act237-c}{14}

	\pregunta{2, 5, 11, 23, \rule{1cm}{0.4pt} }{null}
	\opciones{12}
		{13}
		{14}
		{15}	
	\idpreg{2016-manIPN-rmat-act237-d}{47}	

	\pregunta{3, 4, 7, 11, 18, \rule{1cm}{0.4pt} }{null}
	\opciones{21}
		{29}
		{27}
		{11}	
	\idpreg{2016-manIPN-rmat-act237-k}{29}	

	\pregunta{3, 1, 2, -1, 1, -3, 0, \rule{1cm}{0.4pt} }{null}
	\opciones{1}
		{-1}
		{-5}
		{3}	
	\idpreg{2016-manIPN-rmat-act237-l}{-5}	

	\pregunta{-4, 4, -3, 3, -2, 2, \rule{1cm}{0.4pt}, \rule{1cm}{0.4pt} }{null}
	\opciones{1,-1}
		{-1,1}
		{-5,2}
		{3,1}	
	\idpreg{2016-manIPN-rmat-act237-l}{-1,1}	

	\section*{Sucesiones espaciales}

	\pregunta{}{img/ra_mat/manipn_act241_a.png}
	\idpreg{2016-manIPN-rmat-act241-a}{d}

	\pregunta{}{img/ra_mat/manipn_act241_c.png}
	\idpreg{2016-manIPN-rmat-act241-c}{d}

	\pregunta{}{img/ra_mat/manipn_act241_d.png}
	\idpreg{2016-manIPN-rmat-act241-d}{c}

	\pregunta{}{img/ra_mat/manipn_act241_i.png}
	\idpreg{2016-manIPN-rmat-act241-i}{c}

	\pregunta{}{img/ra_mat/manipn_act241_h.png}
	\idpreg{2016-manIPN-rmat-act241-h}{d}


	\pregunta{Selecciona la opción que continue la serie}{img/ra_mat/conamatbach-rmat-p105-1}
	\opciones[imagen]{img/ra_mat/conamatbach-rmat-p105-1-op-a}
		{img/ra_mat/conamatbach-rmat-p105-1-op-b}
		{img/ra_mat/conamatbach-rmat-p105-1-op-c}
		{img/ra_mat/conamatbach-rmat-p105-1-op-d}
	\idpreg{2016-conamatbach-rmat-p105-1}{c}	

	\pregunta{Selecciona la opción que contenga el término que sigue en está sucesión}{img/ra_mat/conamatbach-rmat-p105-3}
	\opciones[imagen]{img/ra_mat/conamatbach-rmat-p105-3-op-a}
		{img/ra_mat/conamatbach-rmat-p105-3-op-b}
		{img/ra_mat/conamatbach-rmat-p105-3-op-c}
		{img/ra_mat/conamatbach-rmat-p105-3-op-d}
	\idpreg{2016-conamatbach-rmat-p105-3}{a}	

	\pregunta{Selecciona la opción que contenga el término que sigue en está sucesión}{img/ra_mat/conamatbach-rmat-p106-6}
	\opciones[imagen]{img/ra_mat/conamatbach-rmat-p106-6-op-a}
		{img/ra_mat/conamatbach-rmat-p106-6-op-b}
		{img/ra_mat/conamatbach-rmat-p106-6-op-c}
		{img/ra_mat/conamatbach-rmat-p106-6-op-d}
	\idpreg{2016-conamatbach-rmat-p106-6}{a}

	\section*{Imaginación espacial}

	\pregunta{¿Cuál es el área de la parte sombreada?}{img/ra_mat/manipn_act243_b.png}
	\opciones{20 $cm^2$}
		{24 $cm^2$}
		{26 $cm^2$}
		{28 $cm^2$}
	\idpreg{2016-manIPN-rmat-act243-b}{b}	

	\pregunta{¿Cuál de las siguientes figuras giradas no es igual que la de muestra}{img/ra_mat/manipn_act243_e.png}
	\opciones[imagen]{img/ra_mat/manipn_act243_e_op_a.png}
		{img/ra_mat/manipn_act243_e_op_b.png}
		{img/ra_mat/manipn_act243_e_op_c.png}
		{img/ra_mat/manipn_act243_e_op_d.png}
	\idpreg{2016-manIPN-rmat-act243-e}{c}

	\pregunta{Si se realiza un giro a la derecha,¿Cuál de las siguientes figuras corresponde a ésta? Si la figura es la que se muestra a continuación}{img/ra_mat/conamatbach-rmat-p108-19}
	\opciones[imagen]{img/ra_mat/conamatbach-rmat-p108-19-op-a}
		{img/ra_mat/conamatbach-rmat-p108-19-op-b}
		{img/ra_mat/conamatbach-rmat-p108-19-op-c}
		{img/ra_mat/conamatbach-rmat-p108-19-op-d}
	\idpreg{2016-conamatbach-rmat-p109-19}{a}	

\end{multicols*}
\end{document}
