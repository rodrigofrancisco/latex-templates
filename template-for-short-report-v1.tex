\documentclass[12pt]{article}
\usepackage[utf8]{inputenc}
\usepackage[spanish]{babel}
\usepackage{graphicx}
\usepackage{anysize}
\usepackage{fancyhdr} 
\usepackage{titlesec}
\usepackage{longtable}
\usepackage{array}                    % Align fix size columns in tables
\usepackage[export]{adjustbox}
\usepackage{listings}
\usepackage{xcolor}

\decimalpoint%

\setlength{\parindent}{0in}

\renewcommand{\familydefault}{\sfdefault}
\titleformat*{\section}{\large\bfseries}
\marginsize{1.6cm}{2.2cm}{1.2cm}{1.5cm}          % {left}{right}{up}{down}
\lstset{upquote=true}                     % For display quotes and double 
                                          % quoutes in a better style

% Defining column content alignment for fix size columns
\newcolumntype{L}[1]{>{\raggedright\let\newline\\\arraybackslash\hspace{0pt}}m{#1}}
\newcolumntype{C}[1]{>{\centering\let\newline\\\arraybackslash\hspace{0pt}}m{#1}}
\newcolumntype{R}[1]{>{\raggedleft\let\newline\\\arraybackslash\hspace{0pt}}m{#1}}  

%%%%%%%%%%%%%%%%%%%%%%%%%%%%%%%%%%%%%%%%%%%%%%%%%%%%%%%%%%%%%%%%%%%%%%%%%%%%%%%
%%%%%%%%%%                        Code style                         %%%%%%%%%%
%%%%%%%%%%%%%%%%%%%%%%%%%%%%%%%%%%%%%%%%%%%%%%%%%%%%%%%%%%%%%%%%%%%%%%%%%%%%%%%

\definecolor{codegreen}{rgb}{0,0.6,0}
\definecolor{codegray}{rgb}{0.5,0.5,0.5}
\definecolor{codepurple}{rgb}{0.58,0,0.82}
\definecolor{backcolour}{rgb}{1,1,1}

\lstdefinestyle{mystyle}{
    backgroundcolor=\color{backcolour},   
    commentstyle=\color{codegreen},
    keywordstyle=\color{magenta},
    numberstyle=\tiny\color{codegray},
    stringstyle=\color{codepurple},
    basicstyle=\ttfamily\footnotesize,
    breakatwhitespace=false,         
    breaklines=true,                 
    captionpos=b,                    
    keepspaces=true,                 
    % numbers=left,                    
    % numbersep=5pt,                  
    showspaces=false,                
    showstringspaces=false,
    showtabs=false,                  
    tabsize=2
}

\lstset{style=mystyle}


%%%%%%%%%%%%%%%%%%%%%%%%%%%%%%%%%%%%%%%%%%%%%%%%%%%%%%%%%%%%%%%%%%%%%%%%%%%%%%%
%%%%%%%%%%                        Header Style                       %%%%%%%%%%
%%%%%%%%%%%%%%%%%%%%%%%%%%%%%%%%%%%%%%%%%%%%%%%%%%%%%%%%%%%%%%%%%%%%%%%%%%%%%%%

\pagestyle{fancy}
\fancyhf{}
\renewcommand{\headrulewidth}{0pt}
% \setlength{\headsep}{0.3in}


%%%%%%%%%%%%%%%%%%%%%%%%%%%%%%%%%%%%%%%%%%%%%%%%%%%%%%%%%%%%%%%%%%%%%%%%%%%%%%%
%%%%%%%%%%              Cover page generator command                 %%%%%%%%%%
%%%%%%%%%%%%%%%%%%%%%%%%%%%%%%%%%%%%%%%%%%%%%%%%%%%%%%%%%%%%%%%%%%%%%%%%%%%%%%%

\newcommand{\coverPage}{
  \begin{minipage}[t]{0.7\linewidth}
      \vspace{-1cm}
      % \large{\textbf{UNAM}} \large{\textbf{FI}} \\
      \normalsize{\textbf{\materia}}\\
      \normalsize{\textbf{Grupo \grupo}}\\
      % \large{\textbf{Profesor:} \profesor}\\ [1.5cm]
      \normalsize{\textbf{Alumno:} \alumno} \\
      \textbf{Fecha:} \fechaEntrega%
  \end{minipage}\hfill
  \begin{minipage}[t]{0.2\linewidth}
      \vspace{-1.2cm}
      \begin{flushright}
          % \includegraphics[width=2cm]{fi-negro}
          \includegraphics[width=2cm]{unam-negro}\\
          % \vspace{10cm}
          \semestre    
      \end{flushright}
  \end{minipage}
  % \vspace{2mm}
  \begin{center}
    \textbf{\actividad: \textit{\titulo}} \\ 
  \end{center}
}

\usepackage{amsmath}
\begin{document}


%%%%%%%%%%%%%%%%%%%%%%%%%%%%%%%%%%%%%%%%%%%%%%%%%%%%%%%%%%%%%%%%%%%%%%%%%%%%%%%
%%%%%%%%%%                Variables definition                       %%%%%%%%%%
%%%%%%%%%%%%%%%%%%%%%%%%%%%%%%%%%%%%%%%%%%%%%%%%%%%%%%%%%%%%%%%%%%%%%%%%%%%%%%%

\graphicspath{ {latex/assets/} {arq-tarea03.assets/} }
\newcommand{\materia}{Organización y arqitectura de computadoras}
\newcommand{\clave}{1867}
\newcommand{\profesor}{M.I. Pedro Ignacio Rincón Gómez}
\newcommand{\grupo}{4}
\newcommand{\semestre}{2021-2}

\newcommand{\alumno}{Francisco Pablo \textsc{Rodrigo}}

\newcommand{\actividad}{Tarea 03}
\newcommand{\titulo}{Direccionamiento de entrada estado}

\newcommand{\fechaEntrega}{07 de marzo 2021}

\coverPage

%%%%%%%%%%%%%%%%%%%%%%%%%%%%%%%%%%%%%%%%%%%%%%%%%%%%%%%%%%%%%%%%%%%%%%%%%%%%%%%
%%%%%%%%%%                        Contents                           %%%%%%%%%%
%%%%%%%%%%%%%%%%%%%%%%%%%%%%%%%%%%%%%%%%%%%%%%%%%%%%%%%%%%%%%%%%%%%%%%%%%%%%%%%

\textit{Diseñe una carta ASM con hasta 16 estados, 4, entradas (X,Y,Z,W) Y
4 salidas (S0,S1,S2,S3) y determine la tabla de verdad por el método
de Direccionamiento por Entrada-Estado.}\\

\begin{center}
  \includegraphics[width=0.8\linewidth]{arq-tarea03.png}
\end{center}

\textbf{Convención}\\

\includegraphics[width=0.1\linewidth]{convecion}

{
  \setlength\tabcolsep{3.5mm}
  \def\arraystretch{2}          % Do not define globally (for that reason we
                                % enclose table inside brackets)
  \begin{longtable}{
    C{0.3cm}
    C{0.3cm}
    C{0.3cm}
    C{0.3cm}|
    C{0.3cm}
    C{0.3cm}
    C{0.3cm}
    C{0.3cm}
    C{0.3cm}
    C{0.3cm}
    C{0.3cm}
    C{0.3cm}
    C{0.3cm}
    C{0.3cm}
    C{0.3cm}
    C{0.3cm}
    C{0.3cm}
    C{0.3cm}
  }
  % Start: Table header 
  \textbf{$P_3$} &
  \textbf{$P_2$} &
  \textbf{$P_1$} &
  \textbf{$P_0$} &
  \textbf{$K_1$} &
  \textbf{$K_0$} &
  \textbf{$V_3$} &
  \textbf{$V_2$} &
  \textbf{$V_1$} &
  \textbf{$V_0$} &
  \textbf{$F_3$} &
  \textbf{$F_2$} &
  \textbf{$F_1$} &
  \textbf{$F_0$} &
  \textbf{$S_3$} &
  \textbf{$S_2$} &
  \textbf{$S_1$} &
  \textbf{$S_0$} 
  \\ \hline
  % End: Table header 
    0 & 0 & 0 & 0 &
    0 & 0 &
    0 & 0 & 0 & 1 &
    0 & 0 & 1 & 1 &
    0 & 0 & 0 & 1
    \\ \hline
    0 & 0 & 0 & 1 &
    * & * &
    0 & 0 & 1 & 0 &
    0 & 0 & 1 & 0 &
    0 & 0 & 0 & 1
    \\ \hline
    0 & 0 & 1 & 0 &
    0 & 1 &
    0 & 1 & 0 & 1 &
    0 & 0 & 0 & 0 &
    0 & 0 & 1 & 0
    \\ \hline
    0 & 0 & 1 & 1 &
    1 & 0 &
    0 & 1 & 0 & 0 &
    0 & 0 & 1 & 1 &
    0 & 1 & 0 & 0
    \\ \hline
    0 & 1 & 0 & 0 &
    1 & 0 &
    0 & 1 & 1 & 0 &
    1 & 0 & 0 & 0 &
    1 & 1 & 0 & 0
    \\ \hline
    0 & 1 & 0 & 1 &
    * & * &
    0 & 1 & 1 & 0 &
    0 & 1 & 1 & 0 &
    0 & 0 & 0 & 0
    \\ \hline
    0 & 1 & 1 & 0 &
    1 & 1 &
    0 & 1 & 0 & 0 &
    0 & 1 & 1 & 1 &
    1 & 0 & 0 & 1
    \\ \hline
    0 & 1 & 1 & 1 &
    0 & 0 &
    0 & 0 & 1 & 0 &
    0 & 1 & 0 & 1 &
    1 & 0 & 1 & 0
    \\ \hline
    1 & 0 & 0 & 0 &
    * & * &
    0 & 1 & 1 & 0 &
    0 & 1 & 1 & 0 &
    0 & 1 & 0 & 0
    \\ \hline
    1 & 0 & 0 & 1 &
    * & * &
    0 & 0 & 0 & 0 &
    0 & 0 & 0 & 0 &
    0 & 0 & 0 & 0
    \\ \hline
    1 & 0 & 1 & * &
    * & * &
    0 & 0 & 0 & 0 &
    0 & 0 & 0 & 0 &
    0 & 0 & 0 & 0
    \\ \hline
    1 & 1 & * & * &
    * & * &
    0 & 0 & 0 & 0 &
    0 & 0 & 0 & 0 &
    0 & 0 & 0 & 0
  \end{longtable}
}

\textit{Determine el número de bits de memoria que se ahorran al
implementar una carta ASM que posee 4 entradas (X,Y,Z,W), 20
estados, 8 salidas (S0-S7), mediante el método de ``direccionamiento
entrada-estado'' respecto al método ``direccionamiento por trayectoria''.}\\

Ahorro = ``direccionamiento por trayectoria'' - ``direccionamiento
entrada-estado''\\ 

%\begin{equation}
  %\text{Ahorra} = $(2^{\text{bits para edos+ entradas}})(bits de liga+salidas)$
%\end{equation}

\includegraphics[width=\linewidth]{parte-02}


\end{document}
