\documentclass[10pt,a4paper]{article}
	\usepackage{../miconfiguracion}
		\configPage

\begin{document}
	\BgThispage
	\makemytitle{Español}{Primer examen}
	\datosalu
	%\marcaagua
		
	%SECCION DE PREGUNTAS
	\begin{multicols*}{2}
	
\pregunta{Elige la función de la lengua que predomina en el siguiente ejemplo.
Luisa, ¿puedes limpiar la mesa y lavar
los trastes por favor?}{null}
\opciones{Metalingüística.}
{Apelativa.}
{Referencial}
{Sintomática}

\pregunta{Identifica el enunciado en el que la
lengua está usada en su función
poética.}{null}
\opciones
{ Era apenas una niña cuando la vi por
primera vez.}
{A las tres en punto moriría un
transeúnte.}
{Las nieves del tiempo platearon mi
sien.}
{Chopin soñó que estaba muerto en el
lago.}

\pregunta{Identifica la forma del discurso que
predomina en el siguiente párrafo:\\
Los primeros prototipos de los platos
biodegradables eran de piedra caliza,
fécula de papa y papel reciclado, pero
se fue perfeccionando la idea hasta que se logró una mezcla de almidón de papa, agua y un polímero orgánico.}{null}
\opciones
{Argumentación.}
{Narración}
{Demostración}
{Descripción}

\pregunta{¿Qué forma del discurso predomina
en el siguiente texto?\\
Hablamos porque tenemos necesidad
de nombrarnos, de afirmar nuestra
libertad y declarar al mundo nuestro
absoluto derecho a existir.\\
Entendemos entonces que somos
seres que existimos por el lenguaje en
tanto seres comunitarios. Individuos
que nacemos y nos relacionamos a
partir de una vida en comunidad.\\
Comunidad y comunicación no sólo
son términos similares, sino también
esencias que caracterizan a los seres
humanos que existen en el lenguaje.\\
Por ello el lenguaje posee una
condición ontológica en el devenir del
hombre histórico.} {null}
\opciones
{Argumentación}
{Descripción}
{Explicación}
{Exposición}

\pregunta{Lee el siguiente texto y contesta las preguntas\\
Ciencia y filosofía\\
La filosofía es la disciplina que por excelencia se ocupa del pensamiento. Los filósofos siguen siendo esos "enamorados de la sabiduría" cuya labor consiste en preguntarse cosas acerca de la vida y el mundo. Los científicos, por el contrario, son vistos como personajes que se ocupan de una labor más práctica: realizar experimentos para poner a prueba sus
hipótesis sobre cómo funciona la naturaleza.
Cuando un científico se enfrenta a una
pregunta que no puede contestar, es frecuente
que responda "eso es muy filosófico", dando a
entender que no se debería perder el tiempo
con ese tipo de temas. Sin embargo, la
relación entre ciencia y filosofía es muy antigua
y estrecha. De hecho, las diferentes ciencias
surgieron históricamente como parte de la
filosofía, y luego fueron independizándose
conforme dejaron de ser disciplinas basadas
primordialmente en la reflexión para
convertirse en actividades centradas en la
experimentación.\\
Existe una rama de la filosofía que aborda
exclusivamente los problemas de la ciencia.
Algunos son los siguientes: ¿Qué tan bien
funcionan las teorías científicas como
representaciones de la realidad? ¿Qué
distingue a la ciencia de otras formas de
conocimiento? ¿Cómo deciden los científicos
abandonar una teoría para adoptar otra? Y,
finalmente, la pregunta de los 64 mil pesos:
¿por qué funciona la ciencia?\\
Porque, a pesar de lo que pudiera pensarse,
no es para nada obvio que la ciencia nos
proporcione conocimiento objetivo de la
naturaleza: sólo nos brinda modelos e
interpretaciones que pueden ser más o menos
correctas o engañosas. De hecho, no puede
demostrarse que la ciencia sea
intrínsecamente superior a otras formas de
conocimiento. Y, sin embargo, los resultados
prácticos que ofrece son incomparablemente
más efectivos que los de cualquier otra forma
de abordar la realidad. Incluso, algunos
filósofos han desarrollado recientemente una epistemología evolucionista que sugiere que
la ciencia es una adaptación de nuestra
especie cuya función es aumentar nuestras
posibilidades de supervivencia: la ciencia como
producto de la evolución.\\
Todo científico debería conocer algo de
filosofía de la ciencia. Desgraciadamente hay
muchos que no sólo no la conocen, sino que la
desprecian o incluso la ven como algo
amenazante. Y es una lástima, porque no se
puede trabajar bien en algo si no se sabe
cómo funciona.\\
El primer párrafo presenta estructura de}{null}
\opciones
{oposición.}
{interrelación.}
{correlación.}
{contraste.}

\pregunta{Selecciona la opción que indica con
más precisión la estructura del texto.}{null}
\opciones
{Presentación del tema, referencias
sobre el tema, argumentos,
conclusión.}
{Introducción al tema, referencias
sobre el tema, datos, conclusión.}
{Presentación del tema, datos,
desarrollo del tema, conclusión.}
{Introducción al tema, desarrollo del
tema, argumentos, conclusión.}

\pregunta{¿Cuál de los siguientes enunciados
sintetiza mejor la idea principal del
texto?}{null}
\opciones
{Los científicos deben conocer la
reflexión que ofrece la filosofía sobre
el quehacer científico; de este modo
podrán realizar mejor su trabajo.}
{Dado que las diferentes ciencias
surgieron históricamente como parte
de la filosofía, debe haber una
complementariedad entre ciencia y
filosofía.}
{Aunque da resultados más prácticos
que otras formas de conocimiento, la
ciencia no debe considerarse la única
forma de explicar la realidad.}
{Los científicos deben comprender
que sin filosofía no hay ciencia; y los
filósofos, que la ciencia llega a
conclusiones y resultados prácticos
muy efectivos.}

\pregunta{Con la expresión eso es muy
filosófico, el científico}{null}
\opciones
{manifiesta su desconocimiento de la
filosofía.}
{acepta que la filosofía tiene su propio
campo de estudio.}
{admite que la ciencia no puede
explicarlo todo.}
{refleja su forma de entender la
filosofía.}

\pregunta{Si las ciencias son actividades
centradas en la experimentación, se
puede inferir que tienen como última
finalidad \ la realidad.}{null}
\opciones
{medir}
{observar}
{explicar}
{clasificar}

\pregunta{Selecciona la opción que presenta un
sujeto tácito o implícito.}{null}
\opciones
{Somos nada frente a la muerte
infausta.}
{Feliz aquél que busca a Dios en sí
mismo.}
{¡Señor!, tiembla mi alma ante tu
grandeza.}
{Yo he subido más alto, mucho más
alto.}

\pregunta{En el siguiente enunciado, identifica la
función de las palabras en
mayúsculas. Considera el contexto.\\
La señora Ramírez vio vagar sobre los labios de los jefes UNA SONRISA.}{null}
\opciones
{Predicado.}
{Objeto directo.}
{Objeto indirecto.}
{Objeto circunstancial.}
\pregunta{Determina la relación que deben
guardar los enunciados en los
siguientes fragmentos. Selecciona el
grupo de conectores que permite
realizar dicha relación.\\
El Banco de México ya es autónomo
\line(1,0){20}debe seguir siéndolo. El
Congreso debe preocuparse por
establecer la relación jurídica y
política con esa institución, \line(1,0){20}
su operación no esté disociada de los
mecanismos de representación
\line(1,0){20}son inherentes a la función
pública.}{null}
\opciones
{por lo que, así, que}
{y, dado que, los cuales}
{así que, para que, pues}
{y, de modo que, que}
\pregunta{Selecciona la opción que complete el
siguiente enunciado, de manera que
concuerde con el uso formal de la
lengua.\\
Tiene afición\line(1,0){20} las ciencias.}{null}
\opciones
{por}
{para}
{hacia}
{sobre}
\pregunta{Selecciona la opción en la cual existe
un error de concordancia.}{null}
\opciones
{Los checoslovacos combatían en las
calles y se oponían a la dictadura.}
{Un sinnúmero de feligreses acudió a
la Villa.}
{La creación de muchas cosas no se
tenían contempladas.}
{El constante flujo y reflujo de divisas
provocó alarma.}
\pregunta{De acuerdo con la relación de
ANALOGÍA que se establece entre las
palabras en mayúsculas del siguiente
enunciado, señala la opción que
presenta una relación semejante.\\
Tu discurso barroco y con digresiones
provocó ABURRIMIENTO en el
auditorio y, en consecuencia, un
generalizado BOSTEZO.}{null}
\opciones
{Ira, agresión.}
{Diversión, sonrisa.}
{Distracción, error.}
{Impaciencia, rebelión.}
\pregunta{Sinónimo de AVATAR.}{null}
\opciones
{Destino.}
{Cambio.}
{Tragedia.}
{Aflicción.}

\pregunta{Elige la opción con las grafías que
completan correctamente el siguiente
enunciado.\\Aquellas cri\line(1,0){05}is y horribles
erup\line(1,0){05}iones del man\line(1,0){05}o carácter de la
sobrina eran tan fuertes como raras.
Se pasaban a veces cinco o seis años
sin que don Ino\line(1,0){05}encio viera a
Remedios convertirse en una furia.}{null}
\opciones
{s, s, z, s}
{c, c, z, c}
{s, c, s, s}
{s, c, s, c}
\pregunta{Elige la opción que presenta
ortografía INCORRECTA.}{null}
\opciones
{Primitivo pagó con un billete húmedo
y viejo.}
{Así se estuvierón sin decir palabra
largo rato.}
{Dijo adiós tocándose el ala del
sombrero con su mano.}
{Vieron que Eugenia se fue por la
vereda muy temprano.}
\pregunta{De acuerdo con el texto, las palabras huésped y reservorio tienen una relación de}{null}
\opciones
{homonimia.}
{sinonimia.}
{analogía.}
{antonimia.}
\pregunta{¿Cuál de los siguientes enunciados es binembre}{null}
\opciones
{Entre las verdes ramas, un ruiseñor cantaba.}
{Un toro furioso con los cuernos en alto.}
{El golpe blando de sus dedos en las tinieblas.}
{Cruces carcomidas con un aire tétrico.}	
	
\end{multicols*}
	
\end{document}