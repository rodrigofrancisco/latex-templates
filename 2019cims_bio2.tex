\documentclass[10pt,a4paper]{article}
	\usepackage{mylib/miconfiguracion}
  		\configPage

\begin{document}
	\BgThispage
	\makemytitle{Examen comipems.}
	\datosalu
		
	%SECCION DE PREGUNTAS
	\begin{multicols*}{2}
\section{Biología.}
\pregunta{Unidad estructural y funcional de los seres vivos.} {null}
		\opciones{Célula.}
				{Gen.}
				{Tejido.}
				{Órgano.}
\pregunta{Según Darwin, la selección natural es} {null}
		\opciones{el proceso por el cual los individuos que sobreviven son los mejores adaptados.}
				{la presencia de variaciones entre individuos de una misma especie.}
				{la diversidad de seres vivos que han aparecido en el transcurso del tiempo.}
				{la forma de explicar,cómo se originó la vida en el planeta.}
\pregunta{A través del desarrollo\linebreak se busca evitar el deterioro del medio ambiente y satisfacer las necesidades de las generaciones presentes y futuras.} {null}
\opciones{global}{científico}{sustentable}{regional}
\pregunta{El estudio de los microorganismos y el uso de vacunas que nos permiten mejorar nuestra calidad de vida son ejemplos de} {null}
		\opciones{conocimientos empíricos y etimilogía.}
				{ciencias duras y analisís.}
				{conocimiento científico y tecnológico.}
				{ciencia pura y observación.}
\pregunta{Partes de las plantas que generalmente realizan la fotosintesis.} {null}
		\opciones{Flores y tallos.}
				{Raíces y tallos.}
				{Tallos y hojas.}
				{Raíces y hojas.}
\pregunta{Proceso que abarca dos momentos:el intercambio de gases (oxígeno y bióxido de carbono)y la transformación de glucosa en energía química (ATP).} {null}
		\opciones{Respiración}
				{Nutrición}
				{Circulación}
				{Digestión}
\pregunta{Los organismos heterótrofos} {null}
		\opciones{utilizan la energía del Sol para sintetizar alimentos.}
				{transforman la energía  de la luz en energía química.}
				{sintetizan su alimento mediante el procdeso de fotosíntesis.}
				{dependen de otros organismos para obtener su alimento.}
\pregunta{¿Cómo se podría prevenir el sobrepeso y la obesidad?} {null}
		\opciones{Consumir principalmente carnes y pescados.}
				{Tomar suplementos alimenticios y mucha agua.}
				{Llevar una dieta con alimentos ricos en proteínas y grasas.}
				{Llevar una dieta balanceada con verduras, frutas y carnes.}
\pregunta{La contaminación por acumulación de bióxido de azufre en la atmosfera es una condición que provoca de manera secundaria} {null}
		\opciones{lluvia ácida.}
				{efecto invernadero.}
				{abundancia de CO2.}
				{degradación de la capa de ozono.}
\pregunta{La producción de gametos se lleva a cabo por} {null}
		\opciones{meiosis.}
				{replicación.}
				{mitosis.}
				{esporulación.}
\pregunta{Anticonceptivo que tiene como ventaja prevenir infecciones de transmisión sexual.} {null}
		\opciones{Diafragma cervical.}
				{Óvulo vaginal.}
				{Dispositivo intrauterino.}
				{Preservativo o condón.}
\pregunta{Los caracteres hereditarios se transmiten a los hijos mediante} {null}
		\opciones{la sangre.}
				{los cromosomas de las células reproductoras.}
				{los líquidos que pasas de madre a hijo durante la gestación.}
				{el nucléolo del óvulo y el espermatozoide.}
\pregunta{Promueve el cambio de las especies, según la teoría de la evolución proúesta por Darwin.}{null}
\opciones{Necesidad interna del cambio.}
         {Auscencia de mutaciones.}
         {Selección natural.}
         {Herencia de caracteres adquiridos.}   
\pregunta{Conciliar el crecimiento económico de la población con la renovabilidad y permanencia de los recursos naturales, tanto en la actualidad como a futuro, es una medida del}{null} 
\opciones{Crecimiento político.}
         {desarrollo sustentable.}
         {desarrollo industrial.}
         {equilibio urbano.}   
\pregunta{La manipulación de los genes en la obtención de transgénicos es un producto de la interacción de}{null} 
\opciones{ciencia y conocimiento empírico.}
         {ciencia y tecnología.}
         {conocimiento empírico y tecnológico.}
         {conocimiento cietífico y ciencia.}   
\pregunta{La fotosíntesis es un proceso metabólico importante porque}{null} 
\opciones{genera oxígeno y glucosa.}
         {genera bioxido de carbono y glucosa.}
         {produce dióxido de carbono y sales minerales.}
         {prodece ATP para otras estructuras celulares.}   
\pregunta{La respiración aerobica se realiza en el organelo celular llamado}{null} 
\opciones{Mitocrondria.}
         {Peroxisoma}
         {Lisosoma.}
         {Cloroplasto.}   
\pregunta{Alimentos como la carne y la leche aportan a nuestra dieta principalmente}{null} 
\opciones{proteínas y minerales.}
         {carbohidratros y vitaminas.}
         {grasas y vitaminas.}
         {minerales y lípidos.}   
\pregunta{Tipo de reproducción biológica cuyo proposito es la reproducción de individuos diferentes progenitor.}{null} 
\opciones{Asexual.}
         {Gemación.}
         {Bipartición.}
         {Sexual.}   
\pregunta{Los anticonceptivos \linebreak presentan mayor porcentaje de eficacia que los \linebreak .}{null} 
\opciones{físicos-hormonales}
         {naturales-hormonales}
         {químicos-maturales}
         {naturales-mecánicos}   
\pregunta{Tipo de agentes que producen el SIDA.}{null} 
\opciones{Protozoarios.}
         {Virus.}
         {Hongos.}
         {Bacterias.}   
\pregunta{El color de ojos, del pelo y la piel son caracteres genéticos que forman parte de del}{null} 
\opciones{genotipo.}
         {gen recesivo.}
         {fenotipo.}
         {cariotipo}
\pregunta{¿Cómo se le llama al conjunto de todas las reacciones bioquímicas en un organismo?}{null}
\opciones{Anabolismo.}
         {Síntesis.}
         {Metabolismo.}
         {Degradación.}
\pregunta{Ordena las etapas del ciclo del carbono a partir de CO2 almacenado en la atmosfera.\\
I. Al morir todos los seres vivos son degradados por los organismos descomponedores, que liberan CO2 a la atmosfera.\\
II.Las plantas incorporan dióxido de carbono para fabricar azúcares.\\
III.Al comer las plantas, los animales herbívoros asimilan azúcares y liberan CO2.\\
IV.El CO2 liberado en la respiración, llega a la atmósfera y forma parte del aire.}{null} 
\opciones{II,IV,I Y III}
         {II,IV,III Y I}
         {II,I,IV Y III}
         {II,III,IV Y I}   
\pregunta{Es considerado uno de los beneficios más comunes del uso de los organismos transgénicos.}{null} 
\opciones{Recuperación de a biodiversidad.}
         {Prevención y diagnóstico de enfermedades.}
         {Producción alternativa de alimentos y medicinas.}
         {Recuperación de especies en extinción.}   
\pregunta{Estableció que la descendencia presenta variaciones y la selacción natural determina la supervivencia de los demás adeptos. }{null} 
\opciones{Lamarck.}
         {Darwin.}
         {Aristóteles.}
         {Cuvier.}   
\pregunta{Atraves del desarrollo \linebreak se busca evitar el deterioro del medio ambiente y satisfacerlas necesidades de las generaciones presentes y futuras.}{null} 
\opciones{Global}
         {Científico}
         {Sustentable}
         {Regional}   
\pregunta{El bióxido de carbono de la atmósfera pasa a los seres vivos por medio de la}{null} 
\opciones{nutrición}
         {respiración}
         {fotosíntesis}
         {absorción}   
\pregunta{Proceso bioquímico que se lleva a vabo en el interior de la célula para obtener energía.}{null} 
\opciones{Difución gaseosa.}
         {Intercambio gaseoso.}
         {Respiración celular.}
         {Nutrición celular.}   
\pregunta{Tipo de respiración que requiere la presencia de oxígeno libre y se realiza en todas las células eucariontes.}{null} 
\opciones{Pulmonar.}
         {Fermentación.}
         {Branquial.}
         {Aerobia.}   
\pregunta{La producción de gametos se lleva a cabo por.}{null} 
\opciones{meiosis.}
         {replicación.}
         {mitosos.}
         {esporulación.}   
\pregunta{Es considerado uno de los beneficios más comunes del uso de organimos transgénicos}{null} 
\opciones{recuperación de la biodiversidad.}
         {prevencioón y diagnóstico de enfermedades.}
         {producción alternativa de alimentos y medicinas.}
         {recuperación de especies en extinción.}       
\pregunta{La \linebreak es la característica de los seres vivos dónde el organismo responde a los estímulos.}{null} 
\opciones{Alimetación.}
         {Irritabilidad.}
         {Nutrición.}
         {Reproducción.}   
\pregunta{Según Darwin la selección natural es}{null} 
\opciones{el proceso por el cual los individuos que sobreviven son los mejor adaptados.}
         {La presencia de variaciones entre los individuos de una misma especie.}
         {la diversidad de seres vivos que aparecido en el transcurso del tiempo.}
         {la forma de explicar como se originó la vida en el planeta.}   
\pregunta{Los factores orográficos que hacen de México un país megadiverso.}{null} 
\opciones{Océanos Pacífico y Atlantico.}
         {Sierra Madre Oriental y Occidental.}
         {Mesa Central y Altiplano mexicano.}
         {Sierra Madre del Sur y Oriental.}   
\pregunta{¿Cuál de los siguientes organismos es heterótrofo}{null} 
\opciones{Hongo}
         {Algodón}
         {Lenteja}
         {Frijol}   
\pregunta{Los caracteres hereditarios se transmiten a los hijos mediante }{null} 
\opciones{la sangre.}
         {los cromosomas de las células reproductoras.}
         {los líquidos que pasan de madre a hijo durante la gestación.}
         {el núcleo del óvulo y el espermatozoide.}   
\pregunta{Tipos de organismos que podrían sobrevivir en el planeta sin necesidad de otros y producen su propio alimento.}{null} 
\opciones{Hongos.}
         {Autótrofos.}
         {Eucariontes}
         {Anaerobios.}   
\pregunta{La producción de gametos se lleva a cabo por}{null} 
\opciones{Meiosis}
         {Replicación}
         {Mitosis}
         {Esporulación.}   
\pregunta{En una relación sexual la medida más práctica para evitar el contagio de VIH es el uso de }{null} 
\opciones{condón}
         {diafragma}
         {óvulos}
         {Lubricantes}   
\pregunta{Proceso que abarca dos momentos:\\
El intercambio de gases (oxígeno y bioxido de carbono) y la transformación de glucosa en energía química (ATP) }{null} 
\opciones{respiración}
         {circulación}
         {Digestión}
         {Excreción}  
\pregunta{A la constitución o patron hereditario de un organismo, contenido en los genes se le conoce como}{null}
\opciones{Fenotipo}
         {Genotipo}
         {Gene}
         {Cromosoma}
         {Autosoma}
\pregunta{En el síndrome \linebreak se presenta un cromosoma adicional en el par 21.}{null} 
 \opciones{De Klinefelter}
         {de Turner}
         {de Down}
         {de Cri-du-chat}
         {nefróstico}     
         \pregunta{Organismos microscópicos carentes de núcleoverdadero, algunos patógenos, se agrupan el en el reino:}{null} 
 \opciones{Fungi}
         {Animal}
         {Plantae}
         {Monera}
         {Protista} 
         \pregunta{¿Cuál de las siguientes hormonas prepara el útero para que se implante el cigoto?}{null} 
 \opciones{folículo estimulante}
         {progesterona}
         {testosterona}
         {luteinizante}
         {prolactina} 
         \pregunta{La ganaderia intenva puede provocar.}{null} 
 \opciones{Óptimo aprovechamiento de pastos naturales.}
         {Selección natural de pastos.}
         {Selectividad de la domesticación de ganado.}
         {Pérdida de la biodoversidad}
         {enriquecimiento del suelo por abono natural.}
          
         \pregunta{Una diferencia entre la célula animal y vegetal es que la animal no tiene.}{null} 
 \opciones{Membrana plasmática.}
         {Membrana nuclear.}
         {Pared celular.}
         {Ribosomas}
         {Centrosoma.}     
\pregunta{Orden de la mitosis celular.} {null}
		\opciones{fase, metafase,anafase y telofase}
				{profase,metafase,telofase,interfase}
				{telofase,anafase,metafase y proface}
				{proface, metafase,anafase y telofase}
\pregunta{Célula aploide del humano.} {null}
		\opciones{espermatozoide.}
				{hepatocito.}
				{neurona.}
				{espermatogonia}
\pregunta{Nombre de la estructura de los cloroplastos dónde se lleva a cabo el ciclo de Kelvin.} {null}
		\opciones{Grana}
				{Tilacoide}
				{Envoltura cloroplástica.}                           
\end{multicols*}
	
\end{document}