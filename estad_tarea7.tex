\documentclass{mylib/reporte}
\usepackage{amsmath}
%\graphicspath{ {img/labdise_pract6/} }
\usepackage{float}

\title{Reporte}
\author{rodrigofranciscopablo }

\subject{Fundamentos de estadística}
\mytitle{Tarea 7}
\mysubTitle{Tipo 2}
\students{Francisco Pablo \textsc{Rodrigo}}
\teacher{MC. Pineda Norman \textsc{Amanda Lolita}}
\group{3}
\deliverDate{1 de abril de 2019}

\begin{document}

\coverPage

\textbf{Construir el siguiente parámetro}

$$\overline{X} - t_{\frac{\alpha}{2},(n-1)}\frac{S_{n-1}}{\sqrt{n}} \leq \mu \leq \overline{X} + t_{\frac{\alpha}{2},(n-1)}\frac{S_{n-1}}{\sqrt{n}}$$

De donde sabemos que
\begin{enumerate}
  \item  $X \sim N(\mu, \sigma)$
  \item $\sigma^2$ desconocida
  \item n $<$ 30
\end{enumerate}
\begin{align*}
  P(L_1 \leq T \leq L_2 ) &= 1 - \alpha \\
  P(L_1 \leq T \leq L_2 ) & = P(-t_{\frac{\alpha}{2}} \leq T \leq t_{\frac{\alpha}{2}} ) \\
  P(L_1 \leq T \leq L_2 ) &= P(-t_{\frac{\alpha}{2}} \leq \dfrac{\overline{X}-\mu}{\frac{S_{n-1}}{\sqrt{n}}} \leq t_{\frac{\alpha}{2}} )\\
  P(L_1 \leq T \leq L_2 ) &= P(-t_{\frac{\alpha}{2}}\frac{S_{n-1}}{\sqrt{n}} \leq \overline{X}-\mu \leq t_{\frac{\alpha}{2}} \frac{S_{n-1}}{\sqrt{n}} )\\
  P(L_1 \leq T \leq L_2 ) &= P(\overline{X} - t_{\frac{\alpha}{2},(n-1)}\frac{S_{n-1}}{\sqrt{n}} \leq \mu \leq \overline{X} + t_{\frac{\alpha}{2},(n-1)}\frac{S_{n-1}}{\sqrt{n}})
\end{align*}


\textbf{Construir el siguiente parámetro}

$$\hat{P} - z_{\frac{\alpha}{2}} \sqrt{\frac{\hat{P}(1-\hat{P})}{n}} \leq p \leq \hat{P} + z_{\frac{\alpha}{2}} \sqrt{\frac{\hat{P}(1-\hat{P})}{n}}$$

De donde sabemos que
\begin{enumerate}
  \item  p Muestra muy geande
  \item n $\geq$ 30
\end{enumerate}
\begin{align*}
  P(L_1 \leq Z \leq L_2 ) &= 1 - \alpha \\
  P(L_1 \leq Z \leq L_2 ) &= P(-z_{\frac{\alpha}{2}} \leq \dfrac{\hat{P}-p}{\sqrt{\frac{\hat{P}(1-\hat{P})}{n}}} \leq z_{\frac{\alpha}{2}})\\
  P(L_1 \leq Z \leq L_2 ) &= P\left(-z_{\frac{\alpha}{2}}\sqrt{\frac{\hat{P}(1-\hat{P})}{n}} \leq \hat{P}-p \leq z_{\frac{\alpha}{2}}\sqrt{\frac{\hat{P}(1-\hat{P})}{n}}\right)\\
  P(L_1 \leq Z \leq L_2 ) &= P\left(\hat{P} - z_{\frac{\alpha}{2}} \sqrt{\frac{\hat{P}(1-\hat{P})}{n}} \leq p \leq \hat{P} + z_{\frac{\alpha}{2}} \sqrt{\frac{\hat{P}(1-\hat{P})}{n}}\right)\\
\end{align*}

\end{document}
