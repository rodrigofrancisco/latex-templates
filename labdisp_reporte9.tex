\documentclass{mylib/reporteConCalif}
\graphicspath{ {img/labdisp_pract9/} }

\title{Reporte}
\author{rodrigofranciscopablo }

\subject{Laboratorio de Dispositivos y circuitos electrónicos}
\mytitle{Reporte de práctica 10}
\mysubTitle{Transistor de efecto de campo (FET) Caracterización}
\students{Francisco Pablo \textsc{Rodrigo}}
\teacher{M.I. Guevara Rodríguez \textsc{Ma. del Socorro}}
\group{8}
\deliverDate{15 de mayo de 2019}
\usepackage{mathtools}
\usepackage{amsmath}
\usepackage{float}
\usepackage{tabu}
\usepackage{subfig}

\begin{document}

\coverPage

%\tableofcontents
%\newpage

\section{Objetivos}

\subsection{General}

Analizar, diseñar circuitos amplificadores de una etapa con transistores de efecto de campo (FET).

\subsection{Particular}

Analizar, simular y caracterizar un FET, para identificar cada una de sus regiones de operación.

\section{Introducción}

El JFET (transistor de efecto de campo de unión) es un tipo de FET que opera con una unión \textit{pn} polarizada en inversa para contral corriente en un canal. Según su estructura, los JFET caen dentro de dos categorías, de canal n o de canal p.

\insertImage{img1}{Representación de un FET}{10}

En la figura observamos que cada extremo del canal n tiene una terminal; el \textbf{drenaje} se encuentra en el extremo superior de la \textbf{fuente} en el inferior. Se difunden dos regiones tipo p en el material tipo n para formar un \textbf{canal} y ambos tipos de regiones p se conectan a la terminal de la \textbf{compuerta}.


\newpage
\section{Previo}

\subsection{Curva de corriente de drenaje contra voltaje compuerta-fuente}

La curva de transconductancia del JFET es una gráfica que relaciona el $V_{GS}$ con el $I_D$

\insertImage{img2}{Relación de VGS con ID}{12}

\subsection{Curva de corriente de drenaje contra voltaje drenaje-fuente}

A continuación se presentan las curvas características del drenaje de JFET

\insertImage{img3}{Familia de curvas de caracterìstica del denaje}{10}



  \newpage
\section{Desarrollo}


\section{Conclusiones}


\end{document}
