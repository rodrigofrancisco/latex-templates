\documentclass[10pt,a4paper]{article}
	\usepackage{../miconfiguracion}
		\configPage

\begin{document}
	\BgThispage
	\makemytitle{Arítmetica}{Tercer examen}
	\datosalu
	%\marcaagua
		
	%SECCION DE PREGUNTAS
	\begin{multicols*}{2}
	
	\pregunta{La simplificación de $\sqrt{-4}$ es}{null}
		\opciones{i}
				{2i}
				{2}
				{-i}
				{-2}
				\pregunta{La simplificación de $\sqrt{-81}$ es}{null}
		\opciones{9i}
				{-9}
				{i}
				{-9i}
				{-1}
				\pregunta{La simplificación de 3 $ \sqrt{-b}^{4}$ es}{null}
		\opciones{3$^{2}$i}
				{b$^{2}$i}
				{3b$^{2}$i}
				{-3b$^{2}$i}
				{3b$^{2}$}
				\pregunta{Simplificación de: $\sqrt{-x^{2}-y^{2}}$ es}{null}
		\opciones{i$\sqrt{x^{2}+y^{2}}$}
				{$\sqrt{x^{2}+y^{2}}$}
				{i-$\sqrt{x^{2}+y^{2}}$}
				{i$\sqrt{x+y^{2}}$}
				{-i$\sqrt{x^{2}+y^{2}}$}				
				\pregunta{La suma de Z$_{1}=\frac{7}{5}-3i$ $Z_{2}=\dfrac{-2}{5}+10i$}{null}
		\opciones{1+7i}
				{1-7i}
				{-1+7i}
				{$\dfrac{9}{2}+7i$}
				{$\dfrac{9}{2}-7i$}
				\pregunta{A un albañil lo contratan por 6 días, junto con su ayudante por \$2400. Si a
su ayudante le pagan \$115 diarios, ¿cuánto gana el albañil al dia?}{null}
		\opciones{\$275}
				{\$150}
				{\$295}
				{\$305}
				{\$285}
				\pregunta{Descifra el mensaje.
El mensaje codificado siguiente indica las características que tiene
cierta misión. El primer sumando indica el día que hay que empezar la
misión; el segundo, el mes; el tercero, el número de espías que intervendrán
en la misión, finalmente el resultado de la expresión multiplicado por mil es
lo que se pagará a cada espía por la misión. Clave
mensaje:\\
$\dfrac{-60}{-4-1}+\dfrac{-17-39}{5-13}$+[(3+4)-(-1-7)]\\ Precaución:Esta pregunta se autodestruirá en 5 segundos.El mensaje dice: }{null}
		\opciones{La misión comienza el 12 de junio; participan 14 espías; cada uno
recibirá \$ 35,000}
				{La misión comienza el 11 de octubre; participan 15 espías;
cada uno recibirá \$ 33,000}
				{La misión comienza el 12 de julio; participan 15 espías; cada
uno recibirá \$ 34,000}
				{La misión comienza el 10 de mayo; participan 14 espías;
cada uno recibirá \$ 23,000}
                {La misión comienza el -$12$ de julio; participan $-14$ espías;cada uno recibirá \$ $-33,000$}				
				\pregunta{Dos hombres se contratan para realizar un trabajo de albañilería por \$600
laborando durante cinco días. Si uno de ellos recibe un pago de \$ 40 diarios
entonces el salario diario del otro trabajador es:}{null}
		\opciones{\$60}
				{\$30}
				{\$70}
				{\$80}
				{\$90}
		\pregunta{Si una cigarra emerge de la tierra cada 12 años y un tipo de oruga tiene
ciclos de vida de 8 años. Si los ciclos de la cigarra y la oruga coinciden en
2008 entonces el próximo año en que coinciden es:}{null}
		\opciones{2012}
				{2020}
				{2018}
				{2032}
				{2104}
		\pregunta{En una tienda de abarrotes, un empleado vende 3/5 de 
			una pieza de jamón de pierna y después 6/8 del resto. 
			¿Cuánto de jamón quedan. si la pieza entera pesa 6.0 Kg?}
			{null}
		\opciones{3.3 kg}
				{2.4 kg}
				{0.60 kg}
				{2.7 kg}
				{0.9 kg}
				\pregunta{En un contenedor de ferrocarril se cargan 120 metros cúbicos 
			de maíz. Si un metro cúbico de maíz pesa $\mathbf{\dfrac{5}{6}}$ 
			de tonelada, ¿cuál es el peso de la carga en el contenedor?}
			{null}
		\opciones{100 toneladas}
				{110 toneladas}
				{120 toneladas}
				{120.83 toneladas}
				{144 toneladas}
				\pregunta{Si un vestido cuesta \$ 347.30 con IVA incluido
			entonces el precio del vestido sin el 15 \% del IVA es:}
			{null}
		\opciones{\$296}
				{\$300}
				{\$295.20}
				{\$302}
				{\$305.20}
				\pregunta{La razón entre los números de programas respecto a las repeticiones
			en T.V. por cable es 2 a 27. Si Carlos contó solamente 8 nuevos programas
			una noche ¿Cuántas repeticiones hubo?}
			{null}
		\opciones{7}
				{4}
				{108}
				{110}
				{62}
				\pregunta{Identifica un número irracional de los siguientes}
			{null}
		\opciones{5.111111111...}
				{4.011011011011}
				{$\sqrt{5}$}
				{3.04}
				{1, 000, 000}
				\pregunta{A las 3 a.m. de un día de invierno se reporta una
			temperaturna de -3 °C, a las 12 del día la temperatura
			ya  es de 15 °C, ¿de cuántos grados fue la diferncia de
			temperaturas?}
			{null}
		\opciones{18 °C}
				{12 °C}
				{5 °C}
				{-12 °C}

		 \pregunta{En la recta real, el número $\mathbf{\dfrac{7}{8}}$
			se encuentra entre los números}
			{null}
		\opciones{$\dfrac{11}{16}$ y $\dfrac{13}{16}$}
				{$\dfrac{15}{16}$ y $\dfrac{17}{16}$}
				{$\dfrac{25}{32}$ y $\dfrac{27}{32}$}
				{$\dfrac{27}{32}$ y $\dfrac{29}{32}$}
				{$\dfrac{53}{64}$ y $\dfrac{55}{64}$}

		\pregunta{La operación $(-3)^2-[|-7|-|6-8|-|-4|]$=}	{null}
		\opciones{-8}
				{2}
				{13}
				{9}
				{8}
	
	
		\pregunta{Determinar el m.c.m de 60,42 y 12 en términos de números primos}
			{null}
		\opciones{$2^2 \times 3 \times 5 \times 7$}
				{$2 \times 3 \times 5 \times 7$}
				{$2 \times 3^2 \times 5 \times 7$}
				{$2 \times 3 \times 5^2 \times 7$}
				{$2 \times 3 \times 5 \times 7^2$}	
	
		\pregunta{Determina el resultado de la siguiente operación:
			$18+12\div6-3\times2$}
			{null}
		\opciones{30}
				{4}
				{-1}
				{14}
				{20}
		\pregunta{Un ejemplo de número irracional es:}
			{null}
		\opciones{$e = 2.718281...$}		
				{2.34343434...}
				{($\sqrt{2})^2$}
				{0.5}
				{$1.\overline{001}$}
	
	\end{multicols*}
	
\end{document}